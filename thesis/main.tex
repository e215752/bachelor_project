\documentclass[a4j,12pt]{jreport}
\usepackage[dvipdfmx]{graphicx}
\usepackage{listings,jvlisting}
\usepackage{mythesis}
\usepackage{multirow}
\usepackage{amssymb}
\usepackage{amsmath}
\usepackage{array}
\usepackage{here}
\usepackage{enumitem}
\setlength{\itemsep}{-1zh}

\title{ペトリネットに基づく量子最適化の効率化\\
Efficient quantum optimization based on Petri nets}
\icon{
		\includegraphics[width=80mm,bb=0 0 666 502]{fig/logo.gif.pdf}
	}
\year{令和6年度 卒業論文}
\belongto{琉球大学工学部工学科\\
知能情報コース}
\author{215752H 上地 涼太  \\ 指導教員 {名嘉村 盛和} }
%%
%% プリアンブルに記述
%% Figure 環境中で Table 環境の見出しを表示・カウンタの操作に必要
%%
\makeatletter
\newcommand{\figcaption}[1]{\def\@captype{figure}\caption{#1}}
\newcommand{\tblcaption}[1]{\def\@captype{table}\caption{#1}}
\makeatother
\setlength\abovecaptionskip{0pt}

\begin{document}

\maketitle
\baselineskip 17pt plus 1pt minus 1pt


\pagenumbering{roman}
\setcounter{page}{0}

\section*{要旨}
%最後に書こうかな
近年,量子最適化技術が最適化問題の解決手段として注目を集めている.特に,量子アニーリングや量子近似最適化アルゴリズム (QAOA) は、組合せ最適化問題に適用される手法として研究が進められている.しかし,QUBO  や Ising モデルへの定式化が困難であり,大規模かつ複雑な制約を持つ問題に対して高品質な解を得るための手法が求められている.

本研究では,ペトリネットを用いた量子最適化の効率化を目的とし,ペトリネットによる問題の構造化およびQUBOモデルに定式化し多様な解を得る.具体的には,多資源フローショップスケジューリング問題 (MRFSSP) を対象とし,ペトリネットモデルを活用することで,スケジューリング問題のQUBOモデルへの変換を容易にし量子アニーリングを用いた最適化の精度を向上させる.

また,モデル変換による変数削減手法を導入し計算の効率化を図る.問題に対して最適化計算を二段階に分けてペトリネットモデルの変換を行い変数の削減をする.モデル変換の前後の比較実験により,待機時間の縮小を目的とした最適解の探索が可能であることを示し,より効果的なスケジューリングが可能であることを確認した.

本研究の結果は,大規模な組合せ最適化問題における量子計算の実用化に向けた一歩となるものであり,今後の展望としては,リバース量子アニーリングの導入や多目的最適化を維持したモデル変換による高品質な解の探索の改善,有効性を確認したい.

\clearpage
\section*{Abstract}
In recent years, quantum optimization techniques have attracted much attention as a means of solving optimization problems. In particular, quantum annealing and quantum approximate optimization algorithms (QAOA) have been studied as methods applied to combinatorial optimization problems. However, the formulation into QUBO and Ising models is difficult, and methods for obtaining high-quality solutions to problems with large and complex constraints are needed.

In addition, a variable reduction method by model transformation is introduced to improve the efficiency of the calculation. The Petri net model is transformed to reduce variables by dividing the optimization calculation into two stages. Comparison experiments before and after the model transformation showed that it is possible to search for the optimal solution to reduce the waiting time, and confirmed that more effective scheduling is possible.

The results of this study are a step toward the practical application of quantum computation in large-scale combinatorial optimization problems, and future prospects include the introduction of reverse quantum annealing and the improvement and validity of the search for high-quality solutions by model transformation while maintaining multi-objective optimization.
\listoffigures		% 図目次
\listoftables		% 表目次

%以下のように、章ごとに個別の tex ファイルを作成して、
% main.tex をコンパイルして確認する。
%章分けは個人で違うので下のフォーマットを参考にして下さい。

% はじめに
\chapter{序論}
\label{chap:introduction}
\pagenumbering{arabic}


\section{背景と目的}
近年,最適化問題に対するアプローチとして量子最適化技術が注目を集めている.特に,量子アニーリングと量子近似最適化アルゴリズム(QAOA)は,量子コンピュータのために設計された最適化アルゴリズムでとして広く認知されている\cite{qaoa}, \cite{quantum_annealing}.量子最適化の特徴は,従来の古典的アルゴリズムでは解決が困難な問題に対して,高速かつ効率的に最適解を求める潜在能力がある点である.一方で,量子最適化計算の技術は進展しているものの,実用化および普及に向けては多くの課題が存在する.特に,最適化問題をQUBO(Quadratic Unconstrained Binary Optimaization)やIsingモデルで定式化するプロセスが容易ではなく実用化が困難である.また,大規模な問題,制約や目的関数の複雑性による解の質の劣化が実用化における障壁となっている.

本研究では,ペトリネット理論に基づくQUBOやIsingモデルの定式化技術の開発を行う.具体的には,ドメイン知識に基づくモデルベースの定式化手法と効率的な探索を実現するための最適化問題の変換技術を組み合わせることにより,大規模かつ複雑な制約を含む問題に対しても高品質な解が得られる枠組みの構築を目指す.さらに,提案する手法の有効性を確認するために,問題の規模を拡大した場合における性能を評価し従来の手法との解の質の比較を行う.

\section{論文の構成}
本論文は,第1章に序論を述べ,第2章で本実験で使用した基礎概念,第3章で本実験で扱う技術を学ぶために実施した関連研究と評価実験に関して記述する.第4章で本実験の効率化の手法に関する詳細を述べ,最後の第5章にてまとめと今後の課題に関して記述する.


% 基礎概念
\chapter{基礎概念}
\label{chap:concept}

\section{ペトリネット}
ペトリネット$PN = (N,M_o)$は,プレース,トランジションの2種類のノードからなる有向2部グラフ$N = (P,T,Pre,Post)$と初期マーキング$M_0$で表される\cite{murata},\cite{cpn}.

$P = {p_1, p_2, ..., p_n}$はプレースの集合,$T = {t_1, t_2, ..., t_n}$はトランジションの集合である.プレースにトークンを配置することにより現在のシステムの状態を表すことができる.$Pre(p,t)$はトランジション$t$に入る入力プレース$p$を接続するアークの重みを表している.$Post(p,t)$はトランジション$t$から出ている出力プレース$p$を接続するアークの重みを表している.各トランジション$t$の入力プレースに重み以上のトークンが配置されている時トランジション$t$は発火可能であると言う.トランジションの発火は事象の生起を表し,トークンの分布変化が起こる.具体的には,トランジションが発火すると入力プレースから重みの分だけトークンが消費され,出力プレースに重みの分だけトークンが生成されることによってシステムの状態が変化する.

\section{量子アニーリング}
量子アニーリングは量子コンピュータを用いた一種の最適化アルゴリズムであり,最適化問題をQUBOモデルやIsingモデルとしてエネルギー関数の形で表現し,エネルギーを最小化することで最適解に近い解を高速に求めることが期待されている.量子アニーリングは,D-Wave社が開発した商用量子アニーリングマシンにより数万量子ビットの計算が可能であり,実用化が進んでいる.さらに,デジタル方式でQUBOモデルやIsingモデルに基づく最適化処理を行う量子インスパイヤードシステムや,GPUを用いてアニーリングを模倣するシステムも提案されており,これらは新たな組み合わせ最適化プラットフォームとして注目を集めている.

QUBOモデルは(\ref{eq:qubo_model})式のように表すことができる.

\begin{equation}
H = \sum_{i,j} Q_{i,j}q_i q_j
\label{eq:qubo_model}
\end{equation}

ここで$Q_{i,j}\in \mathbb{R}$は,バイナリ変数($q_i \in \{0,1\}$)の係数を表している.また,Isingモデルは(\ref{eq:ising_model})式のように表すことができる.

\begin{equation}
H = \sum_{i < j}J_{i,j} s_i s_j + \sum_i h_i s_i
\label{eq:ising_model}
\end{equation}

ここで$J_{i,j}\in \mathbb{R}$は,スピン変数($s_i = \pm 1$)の間の相互作用を表している.また,$h_{i}\in \mathbb{R}$はスピン$s_i$に作用する外部磁場を表している.

\section{多資源フローショップスケジューリング問題}
多資源フローショップスケジューリング問題(MRFSSP)は,複数の工程からなる複数のジョブを処理するスケジューリング問題の一種である\cite{nouri}.各ジョブは複数の工程によって構成され,各工程は指定された共有資源の利用を必要とする.ジョブ内の各工程はタスクと呼ばれる.

本論文で扱っているMRFSSPを以下にまとめる.

\begin{enumerate} 
\item ジョブの数は$N$個,それぞれのジョブは$M$個のタスク(工程に対応)で構成される. 
\item 各マシンの単位時間当たりの稼働コストが予め与えられている.
\item 各マシンの単位作業時間当たりの処理時間が予め与えられている.
\item 各ジョブは$M$個のタスクを決められた順番で処理する必要があり,その順番は全てのジョブで同一である.
\item 各タスクは予め指定されたマシンリソース群の中から一台のマシンを利用して処理がなされる.一度スタートした処理は中断なく完了まで行われる.
\item 各タスクは同じジョブの直前タスクの完了以前には処理は開始できない(先行制約).
\item 各ジョブの全てのタスクに対して,マシンが割り当てられている(完了制約).
\item 任意の二つのタスク間でマシンの競合が発生しない(競合制約).
\end{enumerate}

また,目的関数として次の二つを考える.
\begin{enumerate} 
\setcounter{enumi}{8} 
\item 各タスクのリソースの待ち時間の総和を最小化する.
\item リソースコストの総和を最小化する.
\end{enumerate}

\section{TPE(Tree-structured Parzen Estimator)}
TPEとは,Parzen推定(カーネル密度推定)に基づいてパラメータ最適化を効率的に行うための手法である.Parzen推定は観測データに基づいて目的関数の値が小さい値を出すパラメータを良い解,大きい値を出すパラメータを悪い解としそれらに対する確率分布を学習する.良い解に近いパラメータを選びやすくするためにこれらの確率分布を利用し次に試すパラメータを探索する\cite{osaka}.これにより,無駄な探索を避け効率的に最適解を見つけることができる\cite{james}.


% 実験
\chapter{ペトリネットに基づく変換前の多目的量子最適化}
\label{chap:poordirection}
量子インスパイアード最適化の論文\cite{shinjo}の定式化を参考にMRFSSPの各タスクのリソースコストとタスク間の待機時間の最適化を図る検証を行った.
\section{多資源フローショップスケジューリング問題}

\subsection{ペトリネットモデル}

\begin{figure}[H]
    \centering
    \includegraphics[width=0.8\linewidth, height=8cm]{./images/fsp.png}
    \caption{フローショップシステムのペトリネットモデル}
    \label{fig:fig1}
\end{figure}

MRFSSPに対するペトリネットモデルは,図\ref{fig:fig1}のように表現できる.図\ref{fig:fig1}では,15個のジョブを省略した形で記載している.各ジョブはプレース,トランジションの交互列となるパスで表されている.例えば,Job1に対するパス$P_{1,1}, T_{1,1}, P_{1,2}, ..., P_{1,4}$は,3種類のタスク(工程)$T_{1,1}, T_{1,2}, T_{1,3}$から構成されており,それぞれ,$R_0, R_1, R_2$のマシンリソースを必要としている.3種類のマシンリソースは,カラートークンとしてマシンID,処理速度,マシンコストの情報を含む文字列で表されている.ここでは紙面の都合上詳細な説明は省略する.このペトリネットモデルのようにドメイン知識さえあれば,わずかなペトリネットの記述のルールに基づいてペトリネットモデルを作成することができる.

\subsection{CPNToolsによるモデル生成}
CPNToolsは,カラードペトリネットのモデリングおよびシミュレーションを支援するツールである.本研究では,CPNToolsを用いて生成したペトリネットモデルをXML形式でエクスポートすることで,プレース,アーク,トランジションに関する構造情報および動作情報を取得できる.

プレース,トランジション,アークのXMLファイルの一部を以下に示す.

\lstset{
  basicstyle={\ttfamily},
  identifierstyle={\small},
  commentstyle={\smallitshape},
  keywordstyle={\small\bfseries},
  ndkeywordstyle={\small},
  stringstyle={\small\ttfamily},
  frame={tb},
  breaklines=true,
  columns=[l]{fullflexible},
  numbers=left,
  xrightmargin=0zw,
  xleftmargin=3zw,
  numberstyle={\scriptsize},
  stepnumber=1,
  numbersep=1zw,
  lineskip=-0.5ex
}
\begin{lstlisting}[caption=プレースのXMLファイル,label=cpn_place]
<place id="ID1412324394">
        <posattr x="-284.000000"
                 y="42.000000"/>
        <fillattr colour="White"
                  pattern=""
                  filled="false"/>
        <lineattr colour="Black"
                  thick="1"
                  type="Solid"/>
        <textattr colour="Black"
                  bold="false"/>
        <text>p_11</text>
        <ellipse w="60.000000"
                 h="40.000000"/>
        <token x="-10.000000"
               y="0.000000"/>
        <marking x="0.000000"
                 y="0.000000"
                 hidden="false">
          <snap snap_id="0"
                anchor.horizontal="0"
                anchor.vertical="0"/>
        </marking>
        <type id="ID1412324395">
          <posattr x="-244.000000"
                   y="18.000000"/>
          <fillattr colour="White"
                    pattern="Solid"
                    filled="false"/>
          <lineattr colour="Black"
                    thick="0"
                    type="Solid"/>
          <textattr colour="Black"
                    bold="false"/>
          <text tool="CPN Tools"
                version="4.0.1">UNIT</text>
        </type>
        <initmark id="ID1412324396">
          <posattr x="-255.000000"
                   y="65.000000"/>
          <fillattr colour="White"
                    pattern="Solid"
                    filled="false"/>
          <lineattr colour="Black"
                    thick="0"
                    type="Solid"/>
          <textattr colour="Black"
                    bold="false"/>
          <text tool="CPN Tools"
                version="4.0.1">1`()</text>
        </initmark>
      </place>
\end{lstlisting}

Listing \ref{cpn_place}は,プレースに関する情報であり<place id>はアークとの関連付けに使用するためのユニークなIDである.<text>プレース</text>はプレースのラベルが指定されており人間が理解しやすい形になっている.<initmark id>はトークンのユニークなIDであり<text tool>トークンの数‘()</text>で初期状態でのトークンの数を表している.

\begin{lstlisting}[caption=トランジションのXMLファイル,label=cpn_transition]
<trans id="ID1412324537"
             explicit="false">
        <posattr x="-138.000000"
                 y="42.000000"/>
        <fillattr colour="White"
                  pattern=""
                  filled="false"/>
        <lineattr colour="Black"
                  thick="1"
                  type="solid"/>
        <textattr colour="Black"
                  bold="false"/>
        <text>t_11</text>
        <box w="60.000000"
             h="38.000000"/>
        <binding x="7.200000"
                 y="-3.000000"/>
        <cond id="ID1412324538">
          <posattr x="-177.000000"
                   y="72.000000"/>
          <fillattr colour="White"
                    pattern="Solid"
                    filled="false"/>
          <lineattr colour="Black"
                    thick="0"
                    type="Solid"/>
          <textattr colour="Black"
                    bold="false"/>
          <text tool="CPN Tools"
                version="4.0.1"/>
        </cond>
        <time id="ID1412324539">
          <posattr x="-93.500000"
                   y="72.000000"/>
          <fillattr colour="White"
                    pattern="Solid"
                    filled="false"/>
          <lineattr colour="Black"
                    thick="0"
                    type="Solid"/>
          <textattr colour="Black"
                    bold="false"/>
          <text tool="CPN Tools"
                version="4.0.1"/>
        </time>
        <code id="ID1412324540">
          <posattr x="-73.500000"
                   y="-9.000000"/>
          <fillattr colour="White"
                    pattern="Solid"
                    filled="false"/>
          <lineattr colour="Black"
                    thick="0"
                    type="Solid"/>
          <textattr colour="Black"
                    bold="false"/>
          <text tool="CPN Tools"
                version="4.0.1"/>
        </code>
        <priority id="ID1412324542">
          <posattr x="-206.000000"
                   y="12.000000"/>
          <fillattr colour="White"
                    pattern="Solid"
                    filled="false"/>
          <lineattr colour="Black"
                    thick="0"
                    type="Solid"/>
          <textattr colour="Black"
                    bold="false"/>
          <text tool="CPN Tools"
                version="4.0.1"/>
        </priority>
      </trans>
\end{lstlisting}

Listing \ref{cpn_transition}は,トランジションに関する情報でありプレース同様にプレースとアークの関連付けに使用するためのユニークなIDが付与されている.また,text要素にはトランジションのラベルが指定されている.さらに,<cond>はトランジションの発火条件,<priority>は発火の優先順位,<time>トランジションの処理時間を記載する要素である.ただし,本研究では<cond>,<priority>,および<time>には設定を行っていない.

\begin{lstlisting}[caption=アークのXMLファイル,label=cpn_arc]
<arc id="ID1412324579"
           orientation="PtoT"
           order="1">
        <posattr x="0.000000"
                 y="0.000000"/>
        <fillattr colour="White"
                  pattern=""
                  filled="false"/>
        <lineattr colour="Black"
                  thick="1"
                  type="Solid"/>
        <textattr colour="Black"
                  bold="false"/>
        <arrowattr headsize="1.200000"
                   currentcyckle="2"/>
        <transend idref="ID1412324537"/>
        <placeend idref="ID1412324394"/>
        <annot id="ID1422288249">
          <posattr x="-211.000000"
                   y="53.000000"/>
          <fillattr colour="White"
                    pattern="Solid"
                    filled="false"/>
          <lineattr colour="Black"
                    thick="0"
                    type="Solid"/>
          <textattr colour="Black"
                    bold="false"/>
          <text tool="CPN Tools"
                version="4.0.1">1`()</text>
        </annot>
      </arc>
\end{lstlisting}

Listing \ref{cpn_arc}は,アークに関する情報でありユニークなIDが付与されている.<arc orientation>では,アークの接続方向の指定がされている.PtoTはプレースからトランジション,TtoPはトランジションからプレースの接続を意味する.

上記のXMLファイルから得られた情報を表にまとめる.

\begin{table}[ht]
    \centering
    \caption{XMLファイルから取得した情報}
    \begin{tabular}{>{$}c<{$} p{0.6\linewidth}}
        \hline
        \text{リスト} & \text{取得情報} \\
        \hline
        resource\_m  & マシンリソースのラベルとリソースの複数の性能 \\
        machine\_processing\_time  & 単位時間あたりの処理時間 \\
        machine\_cost & 単位時間あたりのコスト \\
        job  & 各ジョブのタスクを順番に並べた構造\\
        \hline
    \end{tabular}
    \label{variable}
\end{table}

\subsection{QUBO定式化}
基本的なQUBO定式化の方式に基づいてMRFSSPのペトリネットモデルからの定式化を行う.これらの式で用いた変数を表にまとめる.

\begin{table}[ht]
    \centering
    \caption{エネルギー関数内の変数}
    \begin{tabular}{>{$}c<{$} p{0.6\linewidth}}
        \hline
        \text{変数} & \text{定義} \\
        \hline
        rc^r & 単位時間あたりのマシンリソース$r$の使用コスト \\
        fd^r & マシンリソース$r$を使ってタスクを処理する場合の処理時間 \\
        t_{i}^{j} & ジョブ$j$の$i$番目のタスク \\
        x_{k}^{r}(t_{i}^{j}) & 時刻$k$において$t_{i}^{j}$がマシンリソース$r$を使用していれば$x_{k}^{r}(t_{i}^{j})=1$,そうでなければ$0$を示すバイナリ変数 \\
        \hline
    \end{tabular}
    \label{variable}
\end{table}

ペトリネットの発火規則に従うためには,以下のエネルギー関数をゼロにする必要がある.すなわち,$E_{c1}$及び$E_{c2}$ともトランジションズの発火にはその入力プレース全てにトークンが配置されている必要があり,かつ,同じ入力プレースで同じトークンを前提に発火を行うことは禁止されているため,その競合状態にある場合には,一つのトランジションの発火のみに当該トークンが利用される必要がある.このようにペトリネットの発火規則から以下のエネルギー関数が生成できる.

\begin{align} 
E_{c1} &= \sum_{k_1,k_2} \sum_r \sum_{(j_1,j_2)} \sum_i x_{k_1}^{r}(t_{i}^{j_1}) \cdot x_{k_2}^{r}(t_{i}^{j_2}) \label{eqn:c1}\\ 
E_{c2} &= \sum_{k_1,k_2} \sum_{r_1,r_2} \sum_j \sum_i x_{k_1}^{r_1}(t_{i}^{j}) \cdot x_{k_2}^{r_2}(t_{i+1}^{j}) \label{eqn:c2} 
\end{align}

一方,本スケジューリング問題は,全てのジョブが完了するまでタスクの実行計画を作成することであるので,全てのトランジションがただ一度ずつ発火する必要がある.これより,以下のエネルギー関数が生成できる.

\begin{align} 
E_{c3} &= \left( 1 - \sum_k \sum_r \sum_j \sum_i x_{k}^{r}(t_{i}^{j}) \right)^2 \label{eqn:c3} 
\end{align}

目的関数として,リソースの総コストの最小化とタスクの処理の待機時間の最小化となるため,ペトリネットの振る舞いモデルより以下のエネルギー関数が導かれる.

\begin{align} 
E_{o1} &= \sum_k \sum_r \sum_j \sum_i rc^r \cdot fd^r \cdot x_{k}^{r}(t_{i}^{j}) \label{eqn:o1}\\
E_{o2} &= \sum_i \sum_{k_1,k_2} \sum_{r_1,r_2} \sum_j \sum_i \left( k_1 - fd^{r_2} \right) \cdot x_{k_2}^{r_2}(t_{i+1}^{j}) \cdot x_{k_1}^{r_1}(t_{i}^{j}) \label{eqn:o2} 
\end{align}

式(\ref{eqn:c1}), (\ref{eqn:c2}), (\ref{eqn:c3}), (\ref{eqn:o1}), (\ref{eqn:o2})に重み$A$, $B$, $C$, $D$, $E$を乗じてまとめた全体のエネルギー関数は以下のようになる.

\begin{align} 
E = &A \cdot E_{c1} + B \cdot E_{c2} + C \cdot E_{c3} \ + D \cdot E_{o1} + E \cdot E_{o2} 
\end{align}

\section{評価実験}
ペトリネットモデルからQUBOモデルに定式化したエネルギー関数の最適化を行う.本実験では,OpenJijを用いて最適化計算を行う.問題設定は以下に示す.

\begin{enumerate} 
\item ジョブの数は5個. 
\item パラメータA,B,Cは制約の重み.
\item パラメータDはresource costの重み.
\item パラメータEはwaiting timeの重み.
\item 最適化計算をOpenJijで100回行いベイズ最適化でパラメータチューニングを行う.
\item 5. を5回繰り返す.
\end{enumerate}

\begin{figure}[H]
    \centering
    \includegraphics[width=0.8\linewidth, height=8cm]{./images/parato.png}
    \caption{ペトリネット変換前のMRFSSPのパレート解}
    \label{fig:fig2}
\end{figure}

\begin{table}[ht]
    \centering
    \vspace{-0.3cm}
    \caption{各巡回における実行可能解と個数および平均値}
    \begin{tabular}{|c|c|c|c|}
        \hline
         巡回 & 個数 & Resource Costの平均値 & Waiting Timeの平均値 \\
        \hline
        第1巡 & 9 & 312.6 & 47 \\
        \hline
        第2巡 & 8 & 302.4 & 49.4 \\
        \hline
        第3巡 & 6 & 300 & 44.7\\
        \hline
        第4巡 & 2 & 328 & 55.5 \\
        \hline
        第5巡 & 4 & 297.5 & 44.8 \\
        \hline
        全体の平均 & 5.8 & 308.1 & 48.3 \\
        \hline
    \end{tabular}
    \label{tab:before_feasible}
\end{table}

表\ref{tab:before_feasible}は一度の最適化計算で得られた実行解の個数と,それらのResource CostおよびWaiting Timeの平均値を示している.平均値は実行可能解で得られた数で算出している.また,図\ref{fig:fig2}はペトリネット変換前のMRFSSPの最適化計算を行い,得られたパレート解を可視化したグラフになっている.

\begin{figure}[H]
    \centering
    \includegraphics[width=0.8\linewidth, height=8cm]{./images/gantt1.png}
    \caption{5個のジョブに対するスケジュール}
    \label{fig:fig3}
\end{figure}

図\ref{fig:fig3}は,計算結果の一例をガントチャートに表したものである.本ガントチャートでは,色分けによりジョブごとのタスクが区別されており,ジョブ全体の進行状況が視覚的に把握できる.また,各ジョブ内でタスクの工程順序が正確に守られており,依存関係が反映されたスケジュールになっていることが確認できる.さらに,全てのタスクが適切なマシンに割り当てられ,未処理のタスクが存在しないためリソースの競合や制約違反が発生していないことがわかる.

\section{考察}
現時点での結果として得られたパレートフロントの形状から,複数の目的関数がトレードオフの関係になっていることが明らかとなった.具体的には,Waiting Timeを短縮するとResource Costが増加し,逆にResource Costを抑制するとWaiting Timeが増加する傾向が確認される.この結果は,どの目的関数を優先するかによって選択される解が異なることを示しており,パレートフロント上の解を選択することで効率的かつ実用的な意思決定を行えると考えられる.実行可能解の個数が少ない原因としては,各パラメータの範囲に対する制限の厳しさが影響している可能性があると考えられる.また,探索回数を100回に限定しているため,解空間の十分な探索が行われず,実行可能解が少なくあった可能性も示唆される.この結果は,パラメータ範囲の設定や探索回数の増加が解空間の拡大および実行可能解の増加に寄与する可能性があると思われる.


\chapter{大規模な問題におけるペトリネット変換}
\label{main_survey}

\section{効率よく解を探索するためのマシンリソース配置}
量子アニーリングは,変数の数が少ない場合により効率的に最適解を探索できる特性がある.この特性,問題の解探索に必要な量子ビットの数が変数の数に依存するためである.変数が少ないほど解の探索空間が小さくなり最適解に到達しやすくなる.このため,量子アニーリングを用いる場合は,対象問題を可能な限りコンパクトに表現し変数の数を削減することが重要である.特に,大規模問題を効率的に解くためには,ペトリネット変換や問題分割などの手法を用いて問題のサイズを適切に縮小することが必要である.

\begin{figure}[H]
    \centering
    \includegraphics[width=0.8\linewidth, height=8cm]{./images/transformation.png}
    \caption{変数削減のためのペトリネット変換}
    \label{fig:fig2}
\end{figure}

図\ref{fig:fig2}は対象問題のペトリネットモデルを変換した図になる.このモデル変換は以下の二段階のプロセスで構成される.

\begin{enumerate} 
\item リソースの事前割り当てを適用することによりペトリネットモデルを変換する.ペトリネットモデルを変換するためのエネルギー関数が(\ref{eqn:tran})式である.
\begin{align} 
min \sum_{m \ne m_i}^M \left( \sum_t^T a_{m_i,t} \cdot x_{m_i,t} - \sum_t^T a_{m,t} \cdot x_{m,t}\right)^2 + \left(1 - \sum_t^T x_{m,t} \right)^2 \label{eqn:tran} 
\end{align}

\item (\ref{eqn:tran})式を処理時間またはリソースコストを最適化を行うことで効率的なリソース配置をする.
\end{enumerate}

変数の数は$r \times k \times t$(ここで,$r$はマシンリソース数,$k$は制限の最大時刻,$t$はタスク数)で表される.だが,ペトリネットモデルの変換を行うことにより変数の数が$k \times t$に削減される.この削減により,最適化計算の規模が縮小し計算効率が向上する.

\section{評価実験}

比較とかする





%まとめと今後の課題
\chapter{結論}
\section{まとめ}
本研究では,複数のリソースを対象としたスケジューリング問題を組合せ最適化問題としてQUBOモデルで定式化し,量子アニーリングを用いて最適なスケジューリングを探索する実験を行った.また,ベイズ最適化を活用することで,制約条件のパラメータおよび目的関数のパラメータを自動で調整し,多様な解を効率的に得ることが可能であることを示した.

さらに,効率的な解の探索を目的として,最適化計算を二段階に分けペトリネットモデルへの変換を行う手法を提案した.このモデル変換により,問題の変数数を削減し,良質な解を効率的に探索することが可能となった.特定の目的を優先する場合においては,変換前よりも効率的な解を求める可能性を示唆する結果が得られた。

\section{今後の展望}
本研究では,モデル変換を多目的から単目的に変換する手法を採用した.しかし,この変換により特定の目的関数の値が固定されることで解空間が過度に狭まり,探索の多様性が制限された可能性があると考えられる.今後は,多目的性を維持したままモデル変換を行う手法を検討することで解の質のさらなる向上が期待される.

また,解の探索方法としてリバース量子アニーリングを導入することにより実用的で高品質な解を得られる可能性がある.この手法の採用により,探索精度の向上と計算効率の改善が期待され,現実のスケジューリング問題への適用性がさらに高まると考えられる。

% 今後の課題
%\input{future.tex}

% 謝辞
\chapter*{謝辞}
\thispagestyle{empty}

%基本的な内容は以下の通り.参考にしてみて下さい.
%厳密な決まりは無いので,個々人の文体でも構わない.
%GISゼミや英語ゼミに参加した人はその分も入れておく.
%順番は重要なので気を付けるように.(提出前に周りの人に確認してもらう.)
本研究の遂行にあたり,多くの方々にご指導を賜りました.指導教官である名嘉村盛和教授には,御多忙の中,研究の方向性の決定か結果の考察に至るまで貴重な助言をいただきました.深く感謝いたします.また,合同ゼミで貴重な助言をしていただいた仲地孝之教授,城間政司助教にも深く感謝いたします.おかげで国際会議で研究成果を発表することができました.共に研究活動を頑張った仲地研と城間研の同期に感謝いたします.最後に,物心両面で支えてくれた家族に深く感謝いたします.

\hspace{1zw}
\begin{flushright}
 2025年1月\\
 上地 涼太
\end{flushright}

% 参考文献
% 参考文献
\def\line{−\hspace*{-.7zw}−}

\begin{thebibliography}{99}
%\bibitem{*}内の * は各自わかりやすい名前などをつけて、
%論文中には \cite{*} のように使用する。
%これをベースに書き換えた方が楽かも。
%書籍、論文、URLによって若干書き方が異なる。
%URLを載せる人は参考にした年月日を最後に記入すること。

\bibitem{qaoa} Bingzhi Zhang, Akira Sone, and Quntao Zhuang. Quantum computational phase transition in combinatorial problems. \textit{npj Quantum Information}, Vol. 8, No. 1, pp. 87, 2022.
\bibitem{quantum_annealing} Tadashi Kadowaki and Hidetoshi Nishimori. Quantum anneaing in the transverse ising model. \textit{Phys. Rev. E}, Vol. 58, pp. 5355-5363, Nov 1998.
\bibitem{murata} T.Murata. Petri nets: Properties, analysis and applications. \textit{Proceedings of the IEEE}, Vol. 77, No. 4, pp. 541-580, 1989.
\bibitem{cpn} Kurt Jenson, Lars Michael Kristensen, and Lisa Wells. Coloured Petri Nets and CPN Tools for modeling and validation of concurrent systems. \textit{International Journal on Software Tools for Technology Transfer}, Vol. 9, No. 3, pp. 213-254, 2007.
\bibitem{nouri} Houssem Eddine Nouri, Olfa Belkahla Driss, and Khaled Ghedira. Solving the flexible job shop problem by hybrid metahcuristics-based multiagent model. \textit{Journal on Software Tools fot Technology Transfar}, Vol. 9, No. 3, pp. 213-254, 2007.
\bibitem{osaka} 大阪大学医学部 Python会. Parameter Tuning - 大阪大学医学部 Python会. \url{https://oumpy.github.io/blog/2018/09/parameter_tuning.html}.
\bibitem{james} Bergstra, James, Daniel Yamins, and David Cox. Making a science of model search: Hyperparameter optimization in hundreds of dimensions for vision architectures. \textit{International conference on machine learning}, pp. 115-123, 2013.
\bibitem{shinjo} 新城巧也,名嘉村盛和,猪谷宜彦. 信学技報, 段替え作業を考慮したフローショップスケジューリング問題のペトリネットモデリングとQUBO 定式化. vol. 122, no. 78, MSS2022-17, pp. 90-95, 2022 年6 月
\bibitem{openjij} OpenJij.OpenJij - Optimization Library. \url{https://www.openjij.org}.

\end{thebibliography}


% 付録
%\input{appendix.tex}

\end{document}