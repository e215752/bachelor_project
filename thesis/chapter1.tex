\chapter{序論}
\label{chap:introduction}
\pagenumbering{arabic}


\section{背景と目的}
近年,最適化問題に対するアプローチとして量子最適化技術が注目を集めている.特に,量子アニーリングと量子近似最適化アルゴリズム(QAOA)は,量子コンピュータのために設計された最適化アルゴリズムである\cite{qaoa}, \cite{quantum_annealing}.量子最適化計算の技術は進展しているものの,実用化および普及に向けては多くの課題が存在する.特に,最適化問題をQUBO(Quadratic Unconstrained Binary Optimaization)やIsingモデルで定式化するプロセスが容易ではなく実用化が困難である.また,大規模な問題,制約や目的関数の複雑性による解の質の劣化が実用化における障壁となっている.\\
本研究では,ペトリネット理論に基づくQUBOやIsingモデルの定式化の技術の開発を行う.具体的には,ドメイン知識に基づくモデルベースの定式化と効率探索のための最適化問題の変換技術を組み合わせることにより,大規模かつ複雑な制約を含む問題に対しても高品質な解が得られる枠組みの構築を目指す.
\section{論文の構成}
本論文は,第1章に序論を述べ,第2章で本実験で使用した基礎概念,第3章で本実験で扱う技術を学ぶために実施した関連研究と評価実験に関して記述する.第4章で本実験の効率化の手法に関する詳細を述べ,最後の第5章にてまとめと今後の課題に関して記述する.
