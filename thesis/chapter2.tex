\chapter{基礎概念}
\label{chap:concept}

\section{ペトリネット}
ペトリネット$PN = (N,M_o)$は,プレース,トランジションの2種類のノードからなる有向2部グラフ$N = (P,T,Pre,Post)$と初期マーキング$M_0$で表される\cite{murata},\cite{cpn}.

$P = {p_1, p_2, ..., p_n}$はプレースの集合,$T = {t_1, t_2, ..., t_n}$はトランジションの集合である.プレースにトークンを配置することにより現在のシステムの状態を表すことができる.$Pre(p,t)$はトランジション$t$に入る入力プレース$p$を接続するアークの重みを表している.$Post(p,t)$はトランジション$t$から出ている出力プレース$p$を接続するアークの重みを表している.各トランジション$t$の入力プレースに重み以上のトークンが配置されている時トランジション$t$は発火可能であると言う.トランジションの発火は事象の生起を表し,トークンの分布変化が起こる.具体的には,トランジションが発火すると入力プレースから重みの分だけトークンが消費され,出力プレースに重みの分だけトークンが生成されることによってシステムの状態が変化する.

\section{量子アニーリング}
量子アニーリングは量子コンピュータを用いた一種の最適化アルゴリズムであり,最適化問題をQUBOモデルやIsingモデルとしてエネルギー関数の形で表現し,エネルギーを最小化することで最適解に近い解を高速に求めることが期待されている.量子アニーリングは,D-Wave社が開発した商用量子アニーリングマシンにより数万量子ビットの計算が可能であり,実用化が進んでいる.さらに,デジタル方式でQUBOモデルやIsingモデルに基づく最適化処理を行う量子インスパイヤードシステムや,GPUを用いてアニーリングを模倣するシステムも提案されており,これらは新たな組み合わせ最適化プラットフォームとして注目を集めている.

QUBOモデルは(\ref{eq:qubo_model})式のように表すことができる.

\begin{equation}
H = \sum_{i,j} Q_{i,j}q_i q_j
\label{eq:qubo_model}
\end{equation}

ここで$Q_{i,j}\in \mathbb{R}$は,バイナリ変数($q_i \in \{0,1\}$)の係数を表している.また,Isingモデルは(\ref{eq:ising_model})式のように表すことができる.

\begin{equation}
H = \sum_{i < j}J_{i,j} s_i s_j + \sum_i h_i s_i
\label{eq:ising_model}
\end{equation}

ここで$J_{i,j}\in \mathbb{R}$は,スピン変数($s_i = \pm 1$)の間の相互作用を表している.また,$h_{i}\in \mathbb{R}$はスピン$s_i$に作用する外部磁場を表している.

\section{多資源フローショップスケジューリング問題}
多資源フローショップスケジューリング問題(MRFSSP)は,複数の工程からなる複数のジョブを処理するスケジューリング問題の一種である\cite{nouri}.各ジョブは複数の工程によって構成され,各工程は指定された共有資源の利用を必要とする.ジョブ内の各工程はタスクと呼ばれる.

本論文で扱っているMRFSSPを以下にまとめる.

\begin{enumerate} 
\item ジョブの数は$N$個,それぞれのジョブは$M$個のタスク(工程に対応)で構成される. 
\item 各マシンの単位時間当たりの稼働コストが予め与えられている.
\item 各マシンの単位作業時間当たりの処理時間が予め与えられている.
\item 各ジョブは$M$個のタスクを決められた順番で処理する必要があり,その順番は全てのジョブで同一である.
\item 各タスクは予め指定されたマシンリソース群の中から一台のマシンを利用して処理がなされる.一度スタートした処理は中断なく完了まで行われる.
\item 各タスクは同じジョブの直前タスクの完了以前には処理は開始できない(先行制約).
\item 各ジョブの全てのタスクに対して,マシンが割り当てられている(完了制約).
\item 任意の二つのタスク間でマシンの競合が発生しない(競合制約).
\end{enumerate}

また,目的関数として次の二つを考える.
\begin{enumerate} 
\setcounter{enumi}{8} 
\item 各タスクのリソースの待ち時間の総和を最小化する.
\item リソースコストの総和を最小化する.
\end{enumerate}

\section{TPE(Tree-structured Parzen Estimator)}
TPEとは,Parzen推定(カーネル密度推定)に基づいてパラメータ最適化を効率的に行うための手法である.Parzen推定は観測データに基づいて目的関数の値が小さい値を出すパラメータを良い解,大きい値を出すパラメータを悪い解としそれらに対する確率分布を学習する.良い解に近いパラメータを選びやすくするためにこれらの確率分布を利用し次に試すパラメータを探索する\cite{osaka}.これにより,無駄な探索を避け効率的に最適解を見つけることができる\cite{james}.
