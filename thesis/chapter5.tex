\chapter{結論}
\section{まとめ}
本研究では,複数のリソースを対象としたスケジューリング問題を組合せ最適化問題としてQUBOモデルで定式化し,量子アニーリングを用いて最適なスケジューリングを探索する実験を行った.また,ベイズ最適化を活用することで,制約条件のパラメータおよび目的関数のパラメータを自動で調整し,多様な解を効率的に得ることが可能であることを示した.

さらに,効率的な解の探索を目的として,最適化計算を二段階に分けペトリネットモデルへの変換を行う手法を提案した.このモデル変換により,問題の変数数を削減し,良質な解を効率的に探索することが可能となった.特定の目的を優先する場合においては,変換前よりも効率的な解を求める可能性を示唆する結果が得られた。

\section{今後の展望}
本研究では,モデル変換を多目的から単目的に変換する手法を採用した.しかし,この変換により特定の目的関数の値が固定されることで解空間が過度に狭まり,探索の多様性が制限された可能性があると考えられる.今後は,多目的性を維持したままモデル変換を行う手法を検討することで解の質のさらなる向上が期待される.

また,解の探索方法としてリバース量子アニーリングを導入することにより実用的で高品質な解を得られる可能性がある.この手法の採用により,探索精度の向上と計算効率の改善が期待され,現実のスケジューリング問題への適用性がさらに高まると考えられる。
